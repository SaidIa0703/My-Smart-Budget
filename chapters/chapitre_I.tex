\chapter{Présentation personnelle et du projet}

\section{Présentation du rôle et du contexte}

\begin{tcolorbox}[colback=blue!5!white,colframe=blue!60!black,title=\textbf{Mon rôle}]
Je suis \textbf{développeur full-stack}, chargé de concevoir, développer et maintenir des applications web performantes, en assurant la cohérence entre le front-end et le back-end.  
Mon objectif est de transformer des besoins fonctionnels en solutions techniques robustes, maintenables et évolutives.

Dans le cadre de mon alternance, j’occupe un rôle transversal impliquant une collaboration étroite avec le chef de projet, le designer UI/UX et les autres développeurs.  
J’interviens à chaque étape du cycle de développement :
\begin{itemize}
    \item \textbf{Analyse du besoin} : compréhension des attentes utilisateurs et formalisation du cahier des charges ;
    \item \textbf{Conception technique} : définition de l’architecture, choix des technologies, création des modèles de données ;
    \item \textbf{Développement front-end et back-end} : implémentation des fonctionnalités principales et intégration des API ;
    \item \textbf{Tests et validation} : mise en place de tests unitaires et fonctionnels pour assurer la fiabilité de l’application ;
    \item \textbf{Déploiement et maintenance} : conteneurisation avec Docker, déploiement cloud (Render/AWS) et suivi des performances.
\end{itemize}

Ce rôle me permet d’acquérir une vision complète du cycle de vie d’un projet web tout en développant mes compétences techniques et organisationnelles.
\end{tcolorbox}

\begin{tcolorbox}[colback=gray!5!white,colframe=gray!60!black,title=\textbf{Contexte organisationnel}]
Je travaille au sein d’une entreprise spécialisée dans le développement de solutions numériques sur mesure.  
L’équipe technique, composée d’un chef de projet, de deux développeurs full-stack et d’un designer UI/UX, fonctionne selon une organisation agile (Scrum) qui favorise l’autonomie et la collaboration.

L’objectif de l’entreprise est de proposer des applications modernes, intuitives et sécurisées, conçues pour répondre à des problématiques réelles d’efficacité et d’accessibilité.  
Dans ce cadre, mon projet principal, \textbf{My Smart Budget}, a pour mission de répondre à un besoin croissant : aider les utilisateurs à mieux gérer leurs finances personnelles grâce à une solution claire, intelligente et accessible depuis tout support.

\textbf{My Smart Budget} permet aux utilisateurs de :
\begin{itemize}
    \item suivre leurs revenus et dépenses en temps réel ;
    \item catégoriser automatiquement leurs transactions ;
    \item fixer des objectifs d’épargne et recevoir des alertes personnalisées ;
    \item visualiser leurs données financières sous forme de graphiques interactifs.
\end{itemize}

Le projet repose sur une architecture web moderne et sécurisée :  
\textit{React.js} pour le front-end, \textit{Node.js/Express} pour le back-end, \textit{MongoDB} pour le stockage, et \textit{Docker/AWS} pour le déploiement.
\end{tcolorbox}

\begin{tcolorbox}[colback=green!5!white,colframe=green!60!black,title=\textbf{Durée et planification du projet}]
Mon alternance s’étend sur une année complète. Le développement de \textbf{My Smart Budget} s’est déroulé sur six mois selon quatre grandes phases :
\begin{itemize}
    \item \textbf{Phase 1 – Analyse et conception (mois 1)} : étude des besoins utilisateurs, benchmark des solutions existantes et création des maquettes Figma ;
    \item \textbf{Phase 2 – Développement (mois 2 à 4)} : mise en place de l’API REST, développement du front-end, intégration des services externes et mise en place de la base de données ;
    \item \textbf{Phase 3 – Tests et validation (mois 5)} : élaboration des scénarios de tests, validation fonctionnelle et corrections ;
    \item \textbf{Phase 4 – Déploiement et documentation (mois 6)} : conteneurisation, déploiement sur Render/AWS et rédaction de la documentation technique.
\end{itemize}
\end{tcolorbox}

% ========================
% === PROBLÉMATIQUE ===
% ========================

\section{Problématique identifiée}

\begin{tcolorbox}[colback=red!5!white,colframe=red!60!black,title=\textbf{Problématique reformulée}]
\textbf{Cause :} De nombreux utilisateurs, notamment les jeunes actifs et les familles, rencontrent des difficultés à suivre leurs dépenses quotidiennes de manière claire et centralisée.  
Les outils bancaires existants sont souvent complexes, peu personnalisés et centrés sur la comptabilité pure.

\textbf{Conséquence :} Cette complexité décourage les utilisateurs, qui perdent en visibilité sur leurs finances, entraînant un manque de maîtrise budgétaire et des dépassements réguliers.

\textbf{Impact :} L’utilisateur subit une perte de confiance et de contrôle sur sa situation financière.

\textbf{Problématique principale :}  
\textit{Comment concevoir une application simple, visuelle et intelligente permettant à un utilisateur non expert de reprendre le contrôle de son budget, d’anticiper ses dépenses et d’adopter de meilleures habitudes financières ?}
\end{tcolorbox}

% ========================
% === ANALYSE DES GAINS ===
% ========================

\section{Bénéfices métier attendus}

\begin{tcolorbox}[colback=blue!5!white,colframe=blue!60!black,title=\textbf{Analyse des gains et bénéfices attendus}]
Le projet \textbf{My Smart Budget} vise à générer des gains mesurables tant pour les utilisateurs que pour l’entreprise :
\begin{itemize}
    \item \textbf{Gain de productivité :} réduction du temps de saisie des dépenses grâce à l’automatisation et à la catégorisation intelligente (gain estimé de 30\,\%) ;
    \item \textbf{Réduction d’erreurs :} fiabilisation des calculs et des rapports budgétaires grâce à la centralisation des données ;
    \item \textbf{Amélioration de l’expérience utilisateur :} interface simplifiée, accessible et motivante (taux de satisfaction visé : 90\,\%) ;
    \item \textbf{Valorisation de la marque :} démonstration du savoir-faire technique de l’entreprise sur un produit complet et fonctionnel.
\end{itemize}
\end{tcolorbox}

% ========================
% === OBJECTIFS SMART ===
% ========================

\section{Objectifs SMART}

\begin{tcolorbox}[colback=green!5!white,colframe=green!60!black,title=\textbf{Objectifs SMART du projet}]
\begin{itemize}
    \item \textbf{Spécifique :} Fournir une application web permettant de suivre et d’analyser les dépenses personnelles en temps réel via des tableaux de bord et des alertes automatiques ;
    \item \textbf{Mesurable :} Atteindre un taux d’adoption de 70\,\% sur un panel de 20 utilisateurs internes et réduire de 25\,\% les dépassements budgétaires mensuels ;
    \item \textbf{Atteignable :} Développement réalisé par une équipe de trois personnes sur six mois, avec des sprints de deux semaines ;
    \item \textbf{Pertinent :} Répond à un besoin utilisateur réel d’autonomie et de compréhension financière ;
    \item \textbf{Temporel :} Version 1.0 déployée en octobre 2025, version 2.0 planifiée pour décembre 2025 avec retour d’expérience.
\end{itemize}
\end{tcolorbox}

% ========================
% === INDICATEURS ===
% ========================

\section{Indicateurs de succès}

\begin{tcolorbox}[colback=blue!5!white,colframe=blue!60!black,title=\textbf{Indicateurs de performance mesurables}]
\begin{itemize}
    \item \textbf{Taux d’adoption utilisateur :} 70\,\% trois mois après la mise en ligne ;
    \item \textbf{Taux de satisfaction :} supérieur à 90\,\% selon un questionnaire d’évaluation post-utilisation ;
    \item \textbf{Réduction du temps de saisie des dépenses :} 20\,\% en moyenne grâce à l’automatisation ;
    \item \textbf{Réduction des dépassements budgétaires :} 25\,\% en deux mois d’utilisation ;
    \item \textbf{Disponibilité du service :} 99,5\,\% garantie grâce à l’hébergement cloud.
\end{itemize}
\end{tcolorbox}

% ========================
% === DIAGRAMME CONTEXTE ===
% ========================

\section{Diagramme de contexte}

\begin{tcolorbox}[colback=gray!5!white,colframe=gray!60!black,title=\textbf{Description du diagramme}]
Le diagramme suivant illustre les principaux flux d’informations entre les acteurs et les composants techniques de l’application.  
Chaque flèche représente une interaction ou un échange de données entre les différentes parties du système.
\end{tcolorbox}

\begin{figure}[H]
    \centering
    \includegraphics[width=0.9\textwidth]{assets/images/diagramme_context.png}
    \caption{Diagramme de contexte de l’application My Smart Budget — flux d’informations et interactions principales}
    \label{fig:diagramme_contexte}
\end{figure}

\begin{tcolorbox}[colback=gray!10!white,colframe=gray!70!black,title=\textbf{Légende du diagramme}]
\begin{itemize}
    \item Les flèches pleines représentent les flux d’informations (\textit{requêtes, réponses, synchronisations}) ;
    \item Les flèches pointillées représentent les interactions utilisateur (\textit{actions sur l’interface}) ;
    \item Chaque acteur (utilisateur, administrateur, comptable) interagit via l’interface web avec l’API REST, qui gère la logique métier et la base de données MongoDB.
\end{itemize}
\end{tcolorbox}

% ========================
% === VALEUR AJOUTÉE ===
% ========================

\section{Valeur ajoutée du projet}

\begin{tcolorbox}[colback=green!5!white,colframe=green!60!black,title=\textbf{Apports du projet My Smart Budget}]
\begin{itemize}
    \item Simplifie la gestion financière personnelle grâce à l’automatisation et à la visualisation claire des données ;
    \item Réduit les erreurs et le temps de saisie manuelle de 20\,\% ;
    \item Offre une meilleure compréhension des habitudes de dépenses grâce à des rapports détaillés et personnalisés ;
    \item Favorise la prise de décision financière et l’autonomie des utilisateurs ;
    \item Met en valeur mes compétences techniques (React, Node.js, MongoDB, Docker, AWS) et ma capacité à gérer un projet complet en méthodologie Agile.
\end{itemize}
\end{tcolorbox}
% ========================
% === DIAGRAMME DE CLASSES UML ===
% ========================

\section{Diagramme de classes UML}

\begin{tcolorbox}[colback=gray!5!white,colframe=gray!60!black,title=\textbf{Description du diagramme}]
Le diagramme de classes UML présenté ci-dessous modélise la structure interne de l’application \textbf{My Smart Budget}.  
Il illustre les principales entités du système (utilisateurs, transactions, budgets, catégories, alertes, etc.) ainsi que leurs relations.  
Chaque classe correspond à une table logique dans la base de données MongoDB, tandis que les associations définissent les liens entre ces entités.
\end{tcolorbox}

\begin{figure}[H]
    \centering
    \includegraphics[width=0.95\textwidth]{assets/images/cas-utilisation.png}
    \caption{Diagramme de classes UML de l’application My Smart Budget}
    \label{fig:diagramme_classes}
\end{figure}

\begin{tcolorbox}[colback=gray!10!white,colframe=gray!70!black,title=\textbf{Analyse du modèle de données}]
Le modèle repose sur une organisation simple, cohérente et évolutive, pensée pour faciliter les traitements financiers et les analyses statistiques :
\begin{itemize}
    \item \textbf{Utilisateur} : entité principale du système. Chaque utilisateur possède un ou plusieurs comptes, budgets, transactions et objectifs. Les données d’authentification et de profil sont sécurisées.
    \item \textbf{Compte} : regroupe les informations liées aux comptes bancaires ou portefeuilles virtuels. Il contient plusieurs transactions.
    \item \textbf{Transaction} : enregistre les revenus et dépenses de l’utilisateur, en précisant la catégorie, la date, le montant et la description. Chaque transaction est liée à un compte et une catégorie.
    \item \textbf{Catégorie} : permet de classifier les transactions (ex. alimentation, transport, loisirs). Elle sert également à agréger les données dans les tableaux de bord.
    \item \textbf{Budget} : définit des limites de dépenses par catégorie ou par période. Il est associé à un utilisateur et à une catégorie spécifique.
    \item \textbf{Objectif} : représente une cible d’épargne ou de dépenses à atteindre dans un délai défini (ex. “économiser 500 €”).
    \item \textbf{Alerte} : notifie l’utilisateur lorsqu’un seuil budgétaire est atteint ou lorsqu’une transaction inhabituelle est détectée.
\end{itemize}
\end{tcolorbox}

\begin{tcolorbox}[colback=blue!5!white,colframe=blue!60!black,title=\textbf{Relations entre les entités}]
Le modèle repose sur des relations claires et normalisées :
\begin{itemize}
    \item \textbf{Utilisateur 1..* Compte} : un utilisateur peut avoir plusieurs comptes (ex. compte courant, compte épargne) ;
    \item \textbf{Compte 1..* Transaction} : un compte contient plusieurs transactions ;
    \item \textbf{Transaction *..1 Catégorie} : chaque transaction appartient à une seule catégorie, mais une catégorie peut regrouper plusieurs transactions ;
    \item \textbf{Utilisateur 1..* Budget} : un utilisateur peut définir plusieurs budgets, selon les catégories ;
    \item \textbf{Budget 1..1 Catégorie} : un budget est lié à une catégorie unique ;
    \item \textbf{Utilisateur 1..* Objectif} : chaque utilisateur peut planifier plusieurs objectifs financiers ;
    \item \textbf{Budget 1..* Alerte} : lorsqu’un budget dépasse un seuil, une ou plusieurs alertes peuvent être générées.
\end{itemize}
\end{tcolorbox}

\begin{tcolorbox}[colback=green!5!white,colframe=green!60!black,title=\textbf{Synthèse du modèle}]
L’architecture du modèle de données garantit :
\begin{itemize}
    \item une \textbf{extensibilité} facilitant l’ajout de nouvelles fonctionnalités (ex. gestion multi-devise, import bancaire) ;
    \item une \textbf{intégrité des données} grâce aux liens explicites entre les entités ;
    \item une \textbf{optimisation des performances} avec des requêtes structurées pour l’analyse budgétaire ;
    \item une \textbf{maintenance simplifiée} via des relations logiques et des entités bien isolées.
\end{itemize}

Ainsi, le diagramme de classes constitue la base de la conception technique et de la logique applicative de \textbf{My Smart Budget}.
\end{tcolorbox}
% ========================
% === SYNTHÈSE ET PERSPECTIVES ===
% ========================

\section{Synthèse et perspectives d’évolution du projet}

\begin{tcolorbox}[colback=gray!5!white,colframe=gray!60!black,title=\textbf{Synthèse du projet}]
Le projet \textbf{My Smart Budget} s’inscrit dans une démarche de modernisation et de simplification de la gestion financière personnelle.  
Conçu comme une application web intuitive et intelligente, il vise à offrir une expérience utilisateur fluide tout en intégrant des fonctionnalités avancées telles que :
\begin{itemize}
    \item la \textbf{centralisation des transactions} (revenus et dépenses) en temps réel ;
    \item la \textbf{catégorisation automatique} des opérations selon leur nature ;
    \item la \textbf{gestion des budgets} avec alertes intelligentes et seuils configurables ;
    \item la \textbf{visualisation graphique} des tendances financières et des objectifs atteints ;
    \item l’accès sécurisé via un système d’authentification et une architecture conforme aux bonnes pratiques du développement web moderne.
\end{itemize}

Le développement du projet a permis de mettre en œuvre l’ensemble des compétences clés du référentiel CDA :  
de la conception UML à l’intégration front/back, en passant par la gestion de projet agile, la sécurité applicative et la conteneurisation avec Docker.
\end{tcolorbox}

\begin{tcolorbox}[colback=blue!5!white,colframe=blue!60!black,title=\textbf{Enjeux techniques et fonctionnels maîtrisés}]
Ce projet m’a permis de consolider mes compétences dans plusieurs domaines :
\begin{itemize}
    \item \textbf{Front-end :} création d’interfaces réactives et ergonomiques avec \textit{React.js} et \textit{Tailwind CSS} ;
    \item \textbf{Back-end :} mise en place d’une API REST sécurisée sous \textit{Node.js / Express.js} avec validation des données et gestion des erreurs ;
    \item \textbf{Base de données :} modélisation et gestion des collections sous \textit{MongoDB} avec schémas cohérents ;
    \item \textbf{Sécurité et performances :} intégration de l’authentification JWT, limitation des requêtes et cryptage des mots de passe ;
    \item \textbf{Déploiement :} conteneurisation via \textit{Docker}, tests avec Postman et hébergement sur \textit{Render / AWS}.
\end{itemize}

Ces choix technologiques garantissent la fiabilité, la scalabilité et la maintenabilité de l’application.
\end{tcolorbox}

\begin{tcolorbox}[colback=green!5!white,colframe=green!60!black,title=\textbf{Perspectives d’évolution}]
Afin d’assurer la continuité du projet et d’enrichir son potentiel fonctionnel, plusieurs pistes d’amélioration sont envisagées pour les versions futures :
\begin{itemize}
    \item \textbf{Synchronisation bancaire automatique} : import direct des transactions via API sécurisée (type Linxo, Budget Insight) ;
    \item \textbf{Application mobile native} : développement d’une version mobile avec \textit{React Native} pour une accessibilité complète ;
    \item \textbf{Gestion multi-devise et multi-profil} : permettre à un utilisateur de gérer plusieurs comptes ou familles de budgets ;
    \item \textbf{Module de prévision intelligente} : ajout d’un algorithme prédictif basé sur l’historique des dépenses ;
    \item \textbf{Partage collaboratif du budget} : option permettant à plusieurs utilisateurs (ex. couple) de suivre un budget commun en temps réel.
\end{itemize}

Ces évolutions permettront de transformer \textbf{My Smart Budget} en un véritable assistant financier personnalisé, combinant simplicité d’usage, puissance analytique et connectivité moderne.
\end{tcolorbox}

\begin{tcolorbox}[colback=yellow!5!white,colframe=yellow!60!black,title=\textbf{Bilan personnel}]
Ce projet représente pour moi une expérience professionnelle formatrice et concrète.  
Il m’a permis de :
\begin{itemize}
    \item renforcer mes compétences techniques en développement full-stack ;
    \item améliorer ma rigueur dans la gestion des versions et la documentation du code ;
    \item développer une approche orientée utilisateur et centrée sur l’expérience client ;
    \item adopter une méthodologie agile efficace, basée sur la planification en sprints et la communication d’équipe ;
    \item prendre conscience de l’importance de la sécurité, de la maintenabilité et de l’évolutivité dans un projet réel.
\end{itemize}

En conclusion, \textbf{My Smart Budget} constitue une première étape solide vers la conception d’applications web à forte valeur ajoutée, combinant technologie, ergonomie et utilité concrète.
\end{tcolorbox}