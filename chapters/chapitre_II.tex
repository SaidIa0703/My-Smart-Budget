\chapter{Cadrage et Cahier des Charges}

\section{Objectifs métier, techniques et pédagogiques}

\subsection{Objectifs métier}

\textbf{My Smart Budget} répond à un besoin concret identifié auprès de plusieurs profils d'utilisateurs : de nombreux étudiants, jeunes actifs et familles peinent à suivre leurs dépenses quotidiennes et à planifier leurs budgets mensuels de manière efficace.

L'objectif métier principal est de \textbf{simplifier la gestion budgétaire personnelle} via une interface claire, automatisée et personnalisable, tout en offrant une vision en temps réel de la situation financière.

\subsubsection{Bénéfices attendus mesurables}

\begin{center}
\begin{tabular}{|p{4cm}|p{4.5cm}|p{3cm}|}
\hline
\textbf{Bénéfice} & \textbf{Indicateur de succès} & \textbf{Objectif} \\
\hline
\textbf{Gain de temps} & Temps moyen de saisie mensuelle & $< 15$ min/mois via import CSV \\
\hline
\textbf{Accessibilité multi-profils} & Taux d'adoption par profil & $\geq 70\,\%$ pour chaque persona \\
\hline
\textbf{Éducation financière} & \% utilisateurs atteignant objectifs & $\geq 60\,\%$ après 3 mois \\
\hline
\textbf{Visibilité financière} & Temps d'accès données clés & $< 5$ secondes (dashboard) \\
\hline
\textbf{Simplification du suivi} & Réduction erreurs budgétaires & $-40\,\%$ vs saisie manuelle \\
\hline
\end{tabular}
\end{center}

\subsection{Objectifs techniques}

Architecture \textbf{3-tiers évolutive} : Front-end (Next.js 14/React 18), Back-end (Node.js 20/NestJS 10), Base de données (PostgreSQL 15 via Prisma 5).

\subsubsection{Exigences de performance (mesurables)}

\begin{center}
\begin{tabular}{|l|p{3.5cm}|p{2.5cm}|p{4.5cm}|}
\hline
\textbf{Critère} & \textbf{Métrique} & \textbf{Objectif} & \textbf{Justification métier} \\
\hline
\multirow{2}{*}{\textbf{Performance}} & Time-to-Render dashboard & $< 2$s (P95) & Expérience utilisateur fluide \\
\cline{2-4}
 & Latence API & $< 500$ms (P95) & Réactivité opérations CRUD \\
\hline
\multirow{3}{*}{\textbf{Sécurité}} & Authentification & JWT + bcrypt (cost $\geq 12$) & Protection données RGPD \\
\cline{2-4}
 & Chiffrement & HTTPS/TLS 1.3 & Confidentialité transactions \\
\cline{2-4}
 & Audit & Logs connexions & Traçabilité conformité \\
\hline
\multirow{2}{*}{\textbf{Maintenabilité}} & Couverture tests & $\geq 60\,\%$ (v1.0) & Réduction bugs production \\
\cline{2-4}
 & Migrations & Schémas versionnés & Évolutivité sans rupture \\
\hline
\multirow{2}{*}{\textbf{Scalabilité}} & Architecture & Stateless API + Docker & Support 100 $\rightarrow$ 10k users \\
\cline{2-4}
 & Cache (v2) & Redis sessions & Réduction charge DB \\
\hline
\multirow{2}{*}{\textbf{Fiabilité}} & Disponibilité & $99{,}5\,\%$ (hors maint.) & Accès continu données \\
\cline{2-4}
 & Sauvegardes & Quotidiennes + tests & RPO $< 24$h, RTO $< 4$h \\
\hline
\end{tabular}
\end{center}

\subsection{Objectifs pédagogiques}

Projet inscrit dans le cadre du \textbf{Titre Professionnel Concepteur Développeur d'Applications (CDA - Niveau 6)}.

\subsubsection{Compétences visées et référentiel}

\begin{center}
\begin{tabular}{|p{4.5cm}|p{5cm}|p{5cm}|}
\hline
\textbf{Bloc de compétences CDA} & \textbf{Compétences développées} & \textbf{Livrables associés} \\
\hline
\textbf{Concevoir et développer des composants d'interface} & Design System, composants React réutilisables, accessibilité (WCAG 2.1 AA) & Pages Auth, Dashboard, formulaires validés \\
\hline
\textbf{Concevoir et développer la persistance des données} & Modélisation Merise (MCD/MLD/MPD), Prisma ORM, migrations versionnées & Schémas DB, documentation MCD, scripts SQL \\
\hline
\textbf{Concevoir et développer une application multicouche répartie} & Architecture 3-tiers, API REST (OpenAPI 3.0), services métier (NestJS) & Documentation architecture, API Swagger, tests E2E \\
\hline
\end{tabular}
\end{center}

\textbf{Méthodologie agile} : Sprints de 2 semaines, backlog GitHub, retrospectives documentées, déploiement continu (CI/CD).

\section{Priorisation des fonctionnalités (Méthode MoSCoW)}

La méthode MoSCoW permet de classer les exigences selon leur criticité pour garantir un MVP (Minimum Viable Product) cohérent et livrable.

\subsection{Tableau de priorisation détaillé}

\begin{center}
\small
\begin{tabular}{|p{2.5cm}|p{4cm}|p{4.5cm}|p{2cm}|}
\hline
\textbf{Priorité} & \textbf{Fonctionnalité} & \textbf{Justification métier} & \textbf{Complexité} \\
\hline
\textbf{Must Have} & Authentification JWT + gestion sessions & Sécurité, protection RGPD, prérequis légal & Moyenne (5j) \\
\hline
\textbf{Must Have} & Transactions CRUD + catégorisation & Cœur fonctionnel du suivi budgétaire & Moyenne (8j) \\
\hline
\textbf{Must Have} & Enveloppes budgétaires (plafonds mensuels) & Différenciateur vs tableurs, contrôle budgétaire & Moyenne (6j) \\
\hline
\textbf{Must Have} & Dashboard synthétique (KPIs + graphiques) & Visibilité immédiate, prise de décision & Élevée (10j) \\
\hline
\textbf{Should Have} & Import CSV standardisé & Gain de temps $\times$10, adoption utilisateurs & Moyenne (5j) \\
\hline
\textbf{Should Have} & Export CSV/JSON (portabilité RGPD) & Conformité Article 20 RGPD, transparence & Faible (3j) \\
\hline
\textbf{Could Have} & Notifications (dépassement enveloppe) & Valeur ajoutée, prévention dépassements & Faible (4j) \\
\hline
\textbf{Could Have} & Gestion multi-comptes bancaires & Réalisme pour utilisateurs multiples banques & Moyenne (5j) \\
\hline
\textbf{Won't Have (v1)} & Agrégation bancaire automatique (DSP2) & Complexe (PSD2, APIs tierces), coût élevé & --- \\
\hline
\textbf{Won't Have (v1)} & OCR de factures/tickets & Technologie ML, précision insuffisante v1 & --- \\
\hline
\textbf{Won't Have (v1)} & Recommandations IA budgétaires & Nécessite historique $> 6$ mois & --- \\
\hline
\end{tabular}
\end{center}

\subsection{Justification des exclusions (Won't Have v1)}

\begin{itemize}
  \item \textbf{Agrégation bancaire} : Nécessite certification PSD2 (6--12 mois), coûts APIs élevés (Plaid/Budget Insight), maintenance complexe. Prévu v2.0 après POC.
  \item \textbf{OCR} : Taux d'erreur élevé sur tickets français (formats variés), coût API externe (\$1.50/1000 requêtes). Évaluation v2.1.
  \item \textbf{IA de recommandations} : Algorithmes ML nécessitant dataset de 1000+ utilisateurs sur 6 mois minimum. Prévu v2.5.
\end{itemize}

\section{Périmètre du MVP (Version 1.0)}

\subsection{Fonctionnalités incluses}

\begin{center}
\small
\begin{tabular}{|p{2.5cm}|p{5.5cm}|p{5.5cm}|}
\hline
\textbf{Module} & \textbf{Fonctionnalités} & \textbf{Livrables techniques} \\
\hline
\textbf{Authentification} & Inscription, Connexion JWT (exp 24h), Déconnexion, Réinit. mot de passe & Module NestJS \texttt{auth}, endpoints REST, tests unitaires \\
\hline
\textbf{Profil utilisateur} & CRUD profil, Modification mot de passe, Suppression compte (RGPD) & Module \texttt{users}, page React \texttt{/profile}, validation Zod \\
\hline
\textbf{Catégories} & 15 catégories prédéfinies, Catégories personnalisées, CRUD & Table \texttt{Category}, API CRUD, composant \texttt{CategoryManager} \\
\hline
\textbf{Enveloppes} & Création avec plafond mensuel, Calcul "reste à vivre", Indicateurs visuels, Historique & Module \texttt{budgets}, calculs temps réel, page \texttt{/envelopes} \\
\hline
\textbf{Transactions} & CRUD complet, Filtres (date, catégorie, montant), Recherche textuelle & Module \texttt{transactions}, API paginée (limit 50), tests E2E \\
\hline
\textbf{Import CSV} & Format gabarit, Validation + détection doublons, Rapport d'import & Endpoint \texttt{/import/csv}, parser Papaparse, page \texttt{/import} \\
\hline
\textbf{Dashboard} & KPIs (solde, dépenses, revenus), Graphiques (camembert, courbes), Filtres temporels & Page \texttt{/dashboard}, composants Recharts, requêtes optimisées \\
\hline
\textbf{Export} & Export CSV et JSON des transactions & Endpoints \texttt{/export/csv} et \texttt{/export/json}, boutons UI \\
\hline
\textbf{Droits d'accès} & Rôle Utilisateur et Admin & Guards NestJS \texttt{RoleGuard}, middleware autorisation \\
\hline
\end{tabular}
\end{center}

\subsection{Fonctionnalités exclues de la v1.0}

\begin{itemize}
  \item Agrégation bancaire automatique
  \item OCR de factures
  \item Recommandations IA
  \item Multi-devises (euro uniquement en v1)
  \item Partage de budgets entre utilisateurs
  \item Application mobile native (PWA responsive uniquement)
\end{itemize}

\subsection{Contraintes d'acceptation globales}

\begin{center}
\begin{tabular}{|p{4cm}|p{4cm}|p{5cm}|}
\hline
\textbf{Critère} & \textbf{Seuil d'acceptation} & \textbf{Méthode de mesure} \\
\hline
Performance dashboard & TTR P95 $< 2$s & Lighthouse, GTmetrix \\
\hline
Fiabilité API & Taux d'erreur $< 1\,\%$ (P95) & Logs applicatifs, monitoring \\
\hline
Disponibilité & $99{,}5\,\%$ (hors maintenance) & Uptime monitoring (UptimeRobot) \\
\hline
Sécurité & 0 vulnérabilité critique (OWASP Top 10) & npm audit, Snyk, revue code \\
\hline
\end{tabular}
\end{center}

\section{Cibles, personae et parties prenantes}

\subsection{Personae détaillés}

\subsubsection{Persona 1 --- Léa, 23 ans, Étudiante en Master}

\textbf{Profil} : 800€/mois (APL + job étudiant), vit en colocation à Lyon.

\paragraph{Objectifs}
\begin{itemize}
  \item Suivre ses dépenses contraintes (loyer 400€, transport 70€, alimentation 200€)
  \item Garder un reste à vivre hebdomadaire de 30€ minimum
  \item Éviter les découverts bancaires (frais 8€/incident)
\end{itemize}

\paragraph{Points de douleur actuels}
\begin{itemize}
  \item Oublie de noter 40\,\% de ses achats quotidiens
  \item Fin de mois difficile (reste $< 20$€ les 5 derniers jours)
  \item Utilise un tableur Excel compliqué et non synchronisé
\end{itemize}

\paragraph{Critères de succès avec My Smart Budget}
\begin{itemize}
  \item Visualiser son reste à vivre en 1 clic ($< 5$s)
  \item Recevoir une alerte 3 jours avant épuisement d'une enveloppe
  \item Réduire le temps de saisie mensuel de 2h à 15 min (import CSV)
\end{itemize}

\paragraph{Scénarios d'usage prioritaires}
\begin{enumerate}
  \item Ajout rapide d'une transaction mobile (course de 15€)
  \item Consultation du reste à vivre avant une sortie
  \item Import CSV mensuel de son relevé bancaire
\end{enumerate}

\subsubsection{Persona 2 --- Marc, 38 ans, Père de famille}

\textbf{Profil} : 4500€/mois (couple biactif), 2 enfants (8 et 12 ans), propriétaire en région parisienne.

\paragraph{Objectifs}
\begin{itemize}
  \item Répartir le budget familial par enveloppes (courses 600€, loisirs 300€, enfants 400€, épargne 500€)
  \item Anticiper les pics de dépenses (rentrée scolaire +800€, vacances +1500€)
  \item Impliquer sa conjointe dans le suivi budgétaire
\end{itemize}

\paragraph{Points de douleur actuels}
\begin{itemize}
  \item Manque de visibilité par catégorie (dépenses enfants dispersées)
  \item Dépassements fréquents sur l'enveloppe "loisirs" (+20\,\% en moyenne)
  \item Aucun outil partagé avec sa conjointe (double saisie fastidieuse)
\end{itemize}

\paragraph{Critères de succès avec My Smart Budget}
\begin{itemize}
  \item Respecter 90\,\% des plafonds mensuels d'enveloppes
  \item Visualiser les dépenses enfants consolidées (école, activités, santé)
  \item Exporter un bilan mensuel en PDF pour discussions de couple
\end{itemize}

\paragraph{Scénarios d'usage prioritaires}
\begin{enumerate}
  \item Création de 8 enveloppes avec plafonds mensuels
  \item Analyse des dépassements du mois précédent
  \item Export CSV pour comptable (déclaration impôts)
\end{enumerate}

\subsection{Parties prenantes}

\subsubsection{Parties prenantes primaires}

\begin{center}
\small
\begin{tabular}{|p{3cm}|p{2cm}|p{4cm}|p{4.5cm}|}
\hline
\textbf{Acteur} & \textbf{Rôle} & \textbf{Besoins} & \textbf{Critères de satisfaction} \\
\hline
\textbf{Abdelghani Saidi} & Porteur projet / Dev CDA & Valider compétences CDA, livrables conformes & Soutenance réussie, projet déployé, doc complète \\
\hline
\textbf{Utilisateurs finaux} & Testeurs MVP (3--5 pers.) & Appli stable, intuitive, sans bugs & SUS Score $\geq 75$, erreurs $< 5\,\%$ \\
\hline
\textbf{Jury CDA} & Évaluateurs certification & Conformité référentiel, qualité technique & Grille CDA : $\geq 12/20$ par bloc \\
\hline
\end{tabular}
\end{center}

\subsubsection{Parties prenantes secondaires}

\begin{center}
\small
\begin{tabular}{|p{3.5cm}|c|c|p{6cm}|}
\hline
\textbf{Acteur} & \textbf{Influence} & \textbf{Intérêt} & \textbf{Stratégie d'engagement} \\
\hline
Hébergeur cloud (OVH/Vercel) & Moyenne & Faible & Contrat SLA, surveillance coûts ($< 50$€/mois v1) \\
\hline
Fournisseurs APIs bancaires (v2) & Moyenne & Moyen & Veille technologique, POC Budget Insight (2026) \\
\hline
GitHub/Docker Hub & Faible & Faible & Utilisation gratuite (repos publics), CI/CD inclus \\
\hline
Communauté open-source (post-v1) & Faible & Moyen & Doc contributeurs, licence MIT, roadmap publique \\
\hline
\end{tabular}
\end{center}

\subsection{Matrice influence/intérêt}

\begin{center}
\begin{tabular}{|p{4.5cm}|p{4cm}|p{5cm}|}
\hline
\textbf{Quadrant} & \textbf{Acteurs} & \textbf{Stratégie} \\
\hline
Forte influence + Fort intérêt & Porteur de projet, Jury CDA & \textbf{Gérer étroitement} : communication hebdo, validation jalons \\
\hline
Forte influence + Faible intérêt & Hébergeur (si indispo) & \textbf{Maintenir satisfait} : monitoring 24/7, SLA respecté \\
\hline
Faible influence + Fort intérêt & Utilisateurs testeurs & \textbf{Tenir informés} : newsletter sprint, changelog, feedback loops \\
\hline
Faible influence + Faible intérêt & Fournisseurs secondaires & \textbf{Surveiller} : veille passive, pas d'actions v1 \\
\hline
\end{tabular}
\end{center}

\section{User Stories et Critères d'Acceptation}

\subsection{User Stories priorisées (P1 $\rightarrow$ P3)}

\begin{center}
\small
\begin{tabular}{|c|c|p{6cm}|c|c|}
\hline
\textbf{ID} & \textbf{P} & \textbf{User Story} & \textbf{Valeur} & \textbf{Effort} \\
\hline
US-01 & P1 & En tant qu'utilisateur, je peux créer un compte et me connecter & Prérequis RGPD & 5 \\
\hline
US-02 & P1 & En tant qu'utilisateur, je peux créer/éditer/supprimer des transactions et les catégoriser & Cœur fonctionnel & 8 \\
\hline
US-03 & P1 & En tant qu'utilisateur, je peux créer des enveloppes et voir mon reste à vivre & Différenciateur & 8 \\
\hline
US-04 & P2 & En tant qu'utilisateur, je peux importer un CSV de transactions & Gain temps x10 & 5 \\
\hline
US-05 & P2 & En tant qu'utilisateur, je vois un dashboard avec graphiques et KPIs & Décision rapide & 13 \\
\hline
US-06 & P2 & En tant qu'utilisateur, je peux exporter mes transactions (CSV/JSON) & Conformité RGPD & 3 \\
\hline
US-07 & P3 & En tant qu'utilisateur, je reçois une notification à 80\,\% du plafond & Prévention & 5 \\
\hline
US-08 & P3 & En tant qu'admin, j'accède au backoffice avec stats globales & Pilotage produit & 8 \\
\hline
\end{tabular}
\end{center}

\subsection{Critères d'acceptation détaillés (Definition of Done)}

\subsubsection{US-01 : Compte \& Connexion}

\paragraph{Critères fonctionnels}
\begin{enumerate}
  \item Formulaire d'inscription avec validations :
    \begin{itemize}
      \item Email unique (format RFC 5322)
      \item Mot de passe $\geq 12$ caractères (1 maj, 1 min, 1 chiffre, 1 spécial)
      \item Confirmation mot de passe identique
    \end{itemize}
  \item Mot de passe haché avec bcrypt (cost factor $\geq 12$)
  \item Token JWT signé (HS256, expiration 24h, payload : userId + rôle)
  \item Redirection automatique vers \texttt{/dashboard} après connexion
  \item Mécanisme anti-brute force : 3 tentatives $\rightarrow$ temporisation 5 min
\end{enumerate}

\paragraph{Critères techniques}
\begin{itemize}
  \item Tests unitaires : 5 cas (inscription OK, email doublon, mot de passe faible, JWT valide, JWT expiré)
  \item Tests E2E : parcours complet inscription $\rightarrow$ connexion $\rightarrow$ page protégée
  \item Latence endpoint \texttt{/register} $< 800$ms (P95)
  \item Logs audit : tentatives échouées + IP (conservation 90j)
\end{itemize}

\paragraph{Critères d'acceptation mesurables}
\begin{itemize}
  \item[$\checkmark$] 100\,\% des connexions avec credentials valides réussissent
  \item[$\checkmark$] 0 token JWT accepté après expiration
  \item[$\checkmark$] Taux d'activation $\geq 80\,\%$ (compte créé $\rightarrow$ 1\textsuperscript{re} connexion dans 7j)
\end{itemize}

\subsubsection{US-02 : Transactions CRUD}

\paragraph{Critères fonctionnels}
\begin{enumerate}
  \item Champs obligatoires : date (ISO 8601), montant (décimal 2 chiffres), catégorie (FK), libellé (max 200 char), type (DEBIT/CREDIT)
  \item Création d'une transaction $< 500$ms (P95)
  \item Liste paginée (50 items/page) avec tri (date DESC) et filtres (catégorie, dates, montant)
  \item Édition conservant historique (champ \texttt{updated\_at})
  \item Suppression avec confirmation modale
  \item Total par catégorie mis à jour instantanément
\end{enumerate}

\paragraph{Critères techniques}
\begin{itemize}
  \item Index DB : \texttt{(userId, date DESC)}, \texttt{(userId, category\_id)}
  \item Validation backend (Zod) + frontend (React Hook Form)
  \item Tests unitaires : CRUD complet (8 cas), calculs totaux (3 cas)
  \item Tests E2E : scénario ajout $\rightarrow$ édition $\rightarrow$ suppression
\end{itemize}

\paragraph{Critères d'acceptation mesurables}
\begin{itemize}
  \item[$\checkmark$] Latence création P95 $< 500$ms
  \item[$\checkmark$] 0 incohérence entre total affiché et somme DB
  \item[$\checkmark$] Taux d'erreur validation $< 2\,\%$
\end{itemize}

\subsubsection{US-03 : Enveloppes Budgétaires}

\paragraph{Critères fonctionnels}
\begin{enumerate}
  \item Création enveloppe : nom (unique), plafond mensuel (€), catégories liées (1 à N)
  \item Calcul "reste à vivre" : $\text{Revenus} - \Sigma(\text{Dépenses enveloppes}) - \Sigma(\text{Dépenses hors env.})$
  \item Indicateurs visuels :
    \begin{itemize}
      \item Vert : consommé $< 80\,\%$
      \item Orange : $80\,\% \leq$ consommé $< 100\,\%$
      \item Rouge : consommé $\geq 100\,\%$
    \end{itemize}
  \item Historique mensuel : graphique évolution sur 12 mois
  \item Export CSV des enveloppes
\end{enumerate}

\paragraph{Critères techniques}
\begin{itemize}
  \item Requête SQL optimisée : agrégation par \texttt{envelope\_id} avec index
  \item Recalcul temps réel : polling 30s (v1)
  \item Tests unitaires : calculs "reste à vivre" (5 cas limites)
\end{itemize}

\paragraph{Critères d'acceptation mesurables}
\begin{itemize}
  \item[$\checkmark$] Temps chargement page \texttt{/envelopes} $< 1{,}5$s (P95)
  \item[$\checkmark$] Précision calculs : 0 écart vs calcul SQL manuel
  \item[$\checkmark$] Adoption : $\geq 70\,\%$ utilisateurs créent $\geq 3$ enveloppes (30j)
\end{itemize}

\subsubsection{US-04 : Import CSV}

\paragraph{Critères fonctionnels}
\begin{enumerate}
  \item Format gabarit : \texttt{date,amount,category,label,type}
  \item Validation fichier :
    \begin{itemize}
      \item Format date : ISO 8601 ou français (DD/MM/YYYY)
      \item Montant : décimal positif
      \item Catégorie : nom exact ou ID existant
      \item Détection doublons : même date + montant + libellé
    \end{itemize}
  \item Performance : import 10\,000 lignes en $< 20$s
  \item Rapport d'import : créées/ignorées/erreurs avec numéros de ligne
\end{enumerate}

\paragraph{Critères techniques}
\begin{itemize}
  \item Parser : Papaparse (UTF-8/Latin1)
  \item Traitement par batchs de 500 lignes
  \item Transaction DB : rollback si $> 10\,\%$ lignes invalides
  \item Tests unitaires : 8 cas (fichier valide, date invalide, doublons, etc.)
\end{itemize}

\paragraph{Critères d'acceptation mesurables}
\begin{itemize}
  \item[$\checkmark$] Taux d'import réussi $> 95\,\%$ (fichiers réels)
  \item[$\checkmark$] Temps import 10k lignes $< 20$s
  \item[$\checkmark$] 0 perte de données si interruption
\end{itemize}

\subsubsection{US-05 : Dashboard Interactif}

\paragraph{Critères fonctionnels}
\begin{enumerate}
  \item KPIs affichés :
    \begin{itemize}
      \item Solde total actuel (€)
      \item Dépenses du mois en cours (€)
      \item Revenus du mois en cours (€)
      \item Reste à vivre estimé (€)
    \end{itemize}
  \item Graphiques :
    \begin{itemize}
      \item Camembert : répartition dépenses par catégorie (top 10)
      \item Courbe : évolution dépenses/revenus sur 12 mois
    \end{itemize}
  \item Filtres temporels : mois, trimestre, année, période personnalisée
  \item Responsive : breakpoints 400px, 768px, 1024px
\end{enumerate}

\paragraph{Critères techniques}
\begin{itemize}
  \item Librairie graphiques : Recharts (composants React natifs)
  \item Optimisation requêtes : agrégations SQL avec \texttt{GROUP BY} + index sur \texttt{date}
  \item Cache côté serveur : memoization (v1) pour données mensuelles (TTL 1h)
  \item Tests E2E : vérification affichage graphiques, filtres fonctionnels
\end{itemize}

\paragraph{Critères d'acceptation mesurables}
\begin{itemize}
  \item[$\checkmark$] Time-to-Render (TTR) $< 2$s (P95, Lighthouse)
  \item[$\checkmark$] Taux d'usage hebdomadaire $> 60\,\%$ (Google Analytics)
  \item[$\checkmark$] 0 erreur JS console sur Chrome 120+, Firefox 120+, Safari 17+
\end{itemize}

\subsubsection{US-06 : Export CSV/JSON}

\paragraph{Critères fonctionnels}
\begin{enumerate}
  \item Export CSV : format compatible Excel/LibreOffice (séparateur \texttt{;}, encodage UTF-8 BOM)
  \item Export JSON : structure conforme Article 20 RGPD (machine-readable)
  \item Période sélectionnable : mois, trimestre, année, tout l'historique
  \item Nom fichier : \texttt{transactions\_2025-11.csv} (inclut période)
\end{enumerate}

\paragraph{Critères techniques}
\begin{itemize}
  \item Génération côté serveur (éviter surcharge client)
  \item Limite : 50\,000 transactions max par export
  \item Tests unitaires : 3 formats (CSV, JSON, période vide)
\end{itemize}

\paragraph{Critères d'acceptation mesurables}
\begin{itemize}
  \item[$\checkmark$] Temps génération $< 5$s pour 10k transactions
  \item[$\checkmark$] 100\,\% des exports téléchargés sont lisibles
  \item[$\checkmark$] Conformité RGPD : export inclut toutes les données personnelles
\end{itemize}

\section{Spécifications Fonctionnelles et Techniques}

\subsection{Répartition Front-end / Back-end}

\begin{center}
\scriptsize
\begin{tabular}{|p{2cm}|p{3.5cm}|p{4cm}|p{4cm}|}
\hline
\textbf{Couche} & \textbf{Technologies} & \textbf{Responsabilités} & \textbf{Livrables v1.0} \\
\hline
\textbf{Front-end} & Next.js 14, React 18, TypeScript, Tailwind CSS, React Hook Form + Zod, Recharts & Interface responsive, Validation formulaires, State management, Graphiques, Accessibilité WCAG 2.1 AA & Pages : \texttt{/auth}, \texttt{/transactions}, \texttt{/envelopes}, \texttt{/dashboard}, \texttt{/profile}, \texttt{/import}. Composants réutilisables. Tests E2E (Playwright) \\
\hline
\textbf{Back-end} & Node.js 20, NestJS 10, TypeScript, Passport (JWT), class-validator, Swagger/OpenAPI & API REST (CRUD), Logique métier, Authentification, Rate limiting (10 req/sec), Logs (Winston) & Modules : \texttt{auth}, \texttt{users}, \texttt{transactions}, \texttt{budgets}, \texttt{categories}. Documentation OpenAPI. Tests unitaires (Jest, couv. $\geq 60\,\%$) \\
\hline
\textbf{Base de données} & PostgreSQL 15, Prisma 5, Prisma Migrate & Schémas versionnés, Contraintes d'intégrité, Index performance, Sauvegardes (pg\_dump) & MCD/MLD/MPD (diagrammes UML + Merise). Scripts migrations. Seeds. Politique rétention (soft delete $< 7$ ans) \\
\hline
\end{tabular}
\end{center}

\subsection{Sécurité, Confidentialité et RGPD}

\subsubsection{Authentification et Chiffrement}

\begin{center}
\begin{tabular}{|p{3.5cm}|p{5cm}|p{5cm}|}
\hline
\textbf{Mesure} & \textbf{Implémentation} & \textbf{Justification} \\
\hline
Authentification & JWT (algorithme HS256, clé 256 bits, exp 24h) & Stateless, scalable, standard industrie \\
\hline
Mots de passe & Bcrypt (cost factor 12, salts automatiques) & Résistance brute-force ($2^{12}$ itérations $\approx 250$ms/hash) \\
\hline
HTTPS & TLS 1.3 obligatoire (Let's Encrypt) & Chiffrement transit, protection MITM \\
\hline
Secrets & Variables d'environnement (\texttt{.env} exclu Git) & Pas de secrets en clair dans le code \\
\hline
\end{tabular}
\end{center}

\subsubsection{Matrice Rôles / Permissions}

\begin{center}
\begin{tabular}{|p{4.5cm}|c|c|c|}
\hline
\textbf{Action} & \textbf{Utilisateur} & \textbf{Admin} & \textbf{Invité} \\
\hline
Voir ses transactions & \checkmark & \checkmark & $\times$ \\
\hline
Modifier ses transactions & \checkmark & \checkmark & $\times$ \\
\hline
Voir transactions autres users & $\times$ & \checkmark (anon.) & $\times$ \\
\hline
Créer enveloppes & \checkmark & \checkmark & $\times$ \\
\hline
Accéder backoffice stats & $\times$ & \checkmark & $\times$ \\
\hline
Supprimer compte utilisateur & \checkmark (son compte) & \checkmark (tous) & $\times$ \\
\hline
Gérer catégories globales & $\times$ & \checkmark & $\times$ \\
\hline
Exporter données (RGPD) & \checkmark & \checkmark & $\times$ \\
\hline
\end{tabular}
\end{center}

\subsubsection{Conformité RGPD}

\paragraph{Droits des utilisateurs (implémentation)}

\begin{center}
\begin{tabular}{|p{4cm}|p{5cm}|p{4.5cm}|}
\hline
\textbf{Droit RGPD} & \textbf{Implémentation technique} & \textbf{Délai de réponse} \\
\hline
\textbf{Droit à l'oubli} (Art. 17) & Endpoint \texttt{DELETE /users/:id} avec suppression en cascade (transactions, enveloppes, catégories perso) & Immédiat (soft delete 30j) \\
\hline
\textbf{Droit à la portabilité} (Art. 20) & Endpoint \texttt{GET /export/all} (JSON structuré : profil + transactions + enveloppes) & $< 5$ secondes \\
\hline
\textbf{Droit d'accès} (Art. 15) & Page \texttt{/profile/data} listant toutes les données stockées & Temps réel \\
\hline
\textbf{Consentement explicite} (Art. 7) & Case à cocher lors de l'inscription ("J'accepte la politique de confidentialité") & Inscription bloquée si refus \\
\hline
\textbf{Journalisation accès} (Art. 30) & Table \texttt{audit\_logs} : \texttt{user\_id}, \texttt{action}, \texttt{ip}, \texttt{timestamp} (conservation 90j) & Temps réel \\
\hline
\end{tabular}
\end{center}

\paragraph{Procédure d'exécution (SOP)}

\begin{enumerate}
  \item \textbf{Suppression compte} :
    \begin{itemize}
      \item Utilisateur clique sur "Supprimer mon compte" (\texttt{/profile/delete})
      \item Confirmation par email (lien sécurisé valide 24h)
      \item Soft delete : \texttt{deleted\_at = NOW()}, données conservées 30j (récupération possible)
      \item Hard delete après 30j : purge définitive par job CRON quotidien
    \end{itemize}
  \item \textbf{Export données} :
    \begin{itemize}
      \item Utilisateur clique sur "Télécharger mes données" (\texttt{/profile/export})
      \item Génération archive ZIP (JSON + CSV) côté serveur
      \item Téléchargement direct (pas de stockage temporaire)
      \item Log de l'action dans \texttt{audit\_logs}
    \end{itemize}
\end{enumerate}

\section{Architecture 3-tiers et Responsabilités}

\subsection{Vue d'ensemble (description textuelle)}

L'application suit une architecture \textbf{3-tiers classique} séparant les responsabilités :

\begin{itemize}
  \item \textbf{Client Web} (navigateur) : Interface utilisateur interactive, validation côté client, gestion du state local (React Context/Zustand).
  \item \textbf{API REST} (NestJS sur Node.js) : Logique métier, authentification JWT, validation des entrées, orchestration des requêtes DB, rate limiting, logs structurés.
  \item \textbf{Base de données} (PostgreSQL) : Persistance des données, contraintes d'intégrité référentielle, index de performance, sauvegardes automatisées.
\end{itemize}

\textbf{Déploiement} : Conteneurs Docker orchestrés via Docker Compose (v1) ou Kubernetes (v2). CI/CD via GitHub Actions (tests automatisés, build, déploiement sur Vercel/OVH).

\textbf{Observabilité minimale (v1)} : Journaux applicatifs centralisés (Winston $\rightarrow$ fichiers), métriques de performance (Lighthouse CI), monitoring uptime (UptimeRobot).

\subsection{Architecture logique (responsabilités détaillées)}

\subsubsection{Couche Présentation (Front-end)}

\begin{itemize}
  \item \textbf{Pages} : Routage App Router Next.js (\texttt{app/auth/page.tsx}, \texttt{app/dashboard/page.tsx}, etc.)
  \item \textbf{Composants} : Découpage atomique (Atomic Design) : atomes (boutons, inputs), molécules (formulaires), organismes (cartes dashboard)
  \item \textbf{Validation UI} : React Hook Form + Zod (schémas partagés avec back-end via package commun)
  \item \textbf{State management} : Zustand pour état global (utilisateur connecté, filtres actifs), Context API pour thèmes/i18n
  \item \textbf{Accessibilité} : Attributs ARIA, navigation clavier, contrastes WCAG 2.1 AA, tests avec axe-core
  \item \textbf{Internationalisation (v2)} : next-i18next (français/anglais), détection automatique locale navigateur
\end{itemize}

\subsubsection{Couche Métier (Back-end)}

\begin{itemize}
  \item \textbf{Modules NestJS} : Architecture modulaire avec injection de dépendances (DI) :
    \begin{itemize}
      \item \texttt{AuthModule} : Inscription, connexion, JWT strategy (Passport), refresh tokens (v2)
      \item \texttt{UsersModule} : CRUD profils, gestion rôles, suppression RGPD
      \item \texttt{TransactionsModule} : CRUD transactions, filtres, agrégations (totaux par catégorie)
      \item \texttt{BudgetsModule} : CRUD enveloppes, calcul "reste à vivre", historiques
      \item \texttt{CategoriesModule} : CRUD catégories (globales + personnalisées)
      \item \texttt{ImportModule} : Parsing CSV, validation, détection doublons, rapports
      \item \texttt{ExportModule} : Génération CSV/JSON, compression ZIP
    \end{itemize}
  \item \textbf{Services métier} : Logique de calcul isolée (testable unitairement) :
    \begin{itemize}
      \item \texttt{BudgetCalculatorService} : Calcul reste à vivre, pourcentages consommation enveloppes
      \item \texttt{TransactionAggregatorService} : Agrégations (sommes par catégorie, évolutions mensuelles)
    \end{itemize}
  \item \textbf{Guards \& Interceptors} : Contrôle d'accès (JWT guard, Role guard), transformation réponses, gestion erreurs globales
  \item \textbf{Rate limiting} : Throttler NestJS (10 req/sec par IP, 100 req/min par utilisateur)
  \item \textbf{Logs} : Winston avec niveaux (error, warn, info, debug), rotation quotidienne, format JSON pour parsing
\end{itemize}

\subsubsection{Couche Données (Base de données)}

\begin{itemize}
  \item \textbf{Schémas Prisma} : Modèles typés (User, Transaction, Budget, Category, AuditLog)
  \item \textbf{Relations} : Foreign keys avec \texttt{ON DELETE CASCADE} (suppression utilisateur $\rightarrow$ purge transactions)
  \item \textbf{Index de performance} :
    \begin{itemize}
      \item \texttt{CREATE INDEX idx\_transactions\_user\_date ON transactions(user\_id, date DESC)}
      \item \texttt{CREATE INDEX idx\_transactions\_category ON transactions(category\_id)}
      \item \texttt{CREATE INDEX idx\_budgets\_user ON budgets(user\_id)}
    \end{itemize}
  \item \textbf{Migrations versionnées} : Prisma Migrate (\texttt{prisma migrate dev}, \texttt{prisma migrate deploy})
  \item \textbf{Seeds} : Données de test (15 catégories prédéfinies, 2 utilisateurs démo, 50 transactions)
  \item \textbf{Sauvegardes} : pg\_dump quotidien (rétention 30j), restauration testée mensuellement
\end{itemize}

\section{Choix Technologiques et Alternatives}

\subsection{Tableau comparatif des choix}

\begin{center}
\scriptsize
\begin{tabular}{|p{2.5cm}|p{2.5cm}|p{3cm}|p{5.5cm}|}
\hline
\textbf{Composant} & \textbf{Choix} & \textbf{Alternative} & \textbf{Raison du choix} \\
\hline
\textbf{Back-end} & NestJS & Express.js & Structure modulaire (DI), décorateurs, tests facilités, support TypeScript natif, CLI puissant \\
\hline
\textbf{Front-end} & Next.js & React pur (Vite) & Routage/SSR intégrés, optimisation images, App Router (RSC), SEO, performances, DX supérieure \\
\hline
\textbf{Base de données} & PostgreSQL & MongoDB & Données transactionnelles relationnelles (ACID), agrégations SQL complexes, contraintes d'intégrité, maturité \\
\hline
\textbf{ORM} & Prisma & TypeORM & Schémas typés auto-générés, migrations déclaratives, Prisma Studio (GUI), performances (requêtes optimisées) \\
\hline
\textbf{Cache/Queue (v2)} & Redis & RabbitMQ & Simplicité d'usage, TTL natifs, incréments atomiques (compteurs), pub/sub pour notifications \\
\hline
\textbf{Graphiques} & Recharts & Chart.js & Intégration React fluide (composants déclaratifs), personnalisation facile, bundle size raisonnable \\
\hline
\textbf{Validation} & Zod & Yup, Joi & Inférence TypeScript automatique, schémas composables, performances (parsing rapide), partage front/back \\
\hline
\textbf{State management} & Zustand & Redux Toolkit & Simplicité (sans boilerplate), devtools intégrés, API minimale, performances (re-renders optimisés) \\
\hline
\textbf{Tests E2E} & Playwright & Cypress & Support multi-navigateurs natif (Chrome, Firefox, Safari), parallélisation, debugging puissant \\
\hline
\textbf{CI/CD} & GitHub Actions & GitLab CI & Gratuit pour repos publics, workflows YAML simples, marketplace d'actions, intégration GitHub native \\
\hline
\textbf{Hébergement front} & Vercel & Netlify & Optimisations Next.js natives, edge functions, analytics intégrés, DX (déploiement auto sur push) \\
\hline
\textbf{Hébergement back/DB} & OVH (VPS) & AWS, Google Cloud & Coûts maîtrisés (20€/mois), souveraineté données (RGPD EU), support français \\
\hline
\end{tabular}
\end{center}

\subsection{Justifications détaillées (choix critiques)}

\subsubsection{PostgreSQL vs MongoDB}

\textbf{Pourquoi PostgreSQL ?}
\begin{itemize}
  \item \textbf{Données transactionnelles} : Les transactions financières nécessitent des garanties ACID (Atomicity, Consistency, Isolation, Durability). PostgreSQL excelle sur ces propriétés.
  \item \textbf{Relations complexes} : Modèle relationnel adapté aux liens User $\leftrightarrow$ Transaction $\leftrightarrow$ Category $\leftrightarrow$ Budget.
  \item \textbf{Agrégations SQL} : Calculs complexes (sommes par catégorie, évolutions temporelles) plus performants en SQL qu'en aggregation pipeline MongoDB.
  \item \textbf{Intégrité référentielle} : Foreign keys avec \texttt{ON DELETE CASCADE} garantissent la cohérence (suppression utilisateur $\rightarrow$ purge automatique transactions).
\end{itemize}

\textbf{MongoDB écarté car} :
\begin{itemize}
  \item Denormalisation nécessaire (duplication données) pour performances $\rightarrow$ risque incohérences
  \item Transactions multi-documents complexes (support depuis v4.0 mais moins mature)
  \item Pas de garantie ACID stricte par défaut
\end{itemize}

\subsubsection{NestJS vs Express.js}

\textbf{Pourquoi NestJS ?}
\begin{itemize}
  \item \textbf{Architecture modulaire} : Découpage clair par domaines métier (AuthModule, TransactionsModule, etc.), réutilisabilité, maintenabilité
  \item \textbf{Injection de dépendances (DI)} : Testabilité maximale (mocks faciles), couplage faible entre composants
  \item \textbf{Décorateurs TypeScript} : Code déclaratif lisible (\texttt{@Get()}, \texttt{@UseGuards()}, \texttt{@Body()})
  \item \textbf{Écosystème riche} : Passport (auth), Swagger (doc auto), class-validator (validation), throttler (rate limiting)
  \item \textbf{Standards} : Conformité aux design patterns (Repository, Service, Controller), onboarding facilité pour nouveaux développeurs
\end{itemize}

\textbf{Express.js écarté car} :
\begin{itemize}
  \item Minimaliste $\rightarrow$ nécessite configuration manuelle (routing, validation, error handling)
  \item Pas de structure imposée $\rightarrow$ divergences architecture entre développeurs
  \item Tests plus complexes (moins d'abstractions pour DI/mocks)
\end{itemize}

\subsubsection{Next.js vs React pur (Vite)}

\textbf{Pourquoi Next.js ?}
\begin{itemize}
  \item \textbf{Routage intégré} : App Router avec layouts, loading states, error boundaries (pas de react-router à configurer)
  \item \textbf{SSR/SSG} : Server-Side Rendering pour SEO (page d'accueil publique), Static Site Generation pour performances
  \item \textbf{Optimisations automatiques} : Images (next/image), fonts (next/font), code splitting, tree shaking
  \item \textbf{API Routes} : Endpoints API co-localisés (utile pour webhooks, génération PDFs en v2)
  \item \textbf{DX supérieure} : Fast Refresh, TypeScript out-of-the-box, devtools intégrés
\end{itemize}

\textbf{React pur (Vite) écarté car} :
\begin{itemize}
  \item Configuration manuelle : routage, bundling, optimisations images
  \item Pas de SSR natif (nécessite framework custom ou Remix/Astro)
  \item SEO limité (CSR uniquement sauf configuration complexe)
\end{itemize}

\section{Contraintes, Risques et Mitigations}

\subsection{Matrice des risques}

\begin{center}
\scriptsize
\begin{tabular}{|p{4cm}|c|c|p{7cm}|}
\hline
\textbf{Risque/Contrainte} & \textbf{Impact} & \textbf{Prob.} & \textbf{Mitigation} \\
\hline
Retard développement back-end & Moyen & Élevée & Sprints courts (2 sem.), feature flags (déploiement progressif), priorité P1 stricte, buffer 20\,\% planning \\
\hline
Failles sécurité (JWT/SQL injection) & Élevé & Moyen & Revue de code obligatoire (PR), tests automatisés (Snyk, npm audit), linters (ESLint security rules), checklist OWASP Top 10 \\
\hline
Charge import CSV (timeout) & Moyen & Moyen & Traitement par batchs (500 lignes), transactions DB atomiques, worker asynchrone (Bull/Redis en v2), index DB optimisés \\
\hline
Complexité responsive (breakpoints) & Moyen & Moyen & Design mobile-first (Tailwind), tests sur appareils réels (BrowserStack), composants adaptatifs (useMediaQuery) \\
\hline
Dépendance hébergeur cloud & Faible & Faible & Abstraction infra (Docker), variables env (12-factor app), sauvegardes externes (S3/Backblaze), documentation déploiement \\
\hline
Dépassement budget hébergement & Faible & Moyen & Monitoring coûts (alerts OVH), optimisation requêtes (cache Redis v2), rate limiting, scaling horizontal différé (v2) \\
\hline
Perte de données (crash DB) & Élevé & Faible & Sauvegardes quotidiennes (pg\_dump), réplication (read replica en v2), tests de restauration mensuels, transactions ACID \\
\hline
Conformité RGPD non respectée & Élevé & Faible & Checklist RGPD exhaustive, endpoints dédiés (export/suppression), logs audit, revue juridique (DPO consulté), doc politique confidentialité \\
\hline
\end{tabular}
\end{center}

\subsection{Plan de contingence (risques critiques)}

\subsubsection{Failles de sécurité découvertes en production}

\textbf{Scénario} : Vulnérabilité critique détectée (ex : JWT mal signé, SQL injection possible).

\textbf{Plan d'action} :
\begin{enumerate}
  \item \textbf{Immédiat (H+0)} : Mise hors ligne temporaire de l'API (page maintenance), notification utilisateurs par email
  \item \textbf{H+2} : Patch de sécurité développé et testé sur environnement staging
  \item \textbf{H+4} : Déploiement du patch en production, tests de non-régression
  \item \textbf{H+6} : Remise en ligne, audit de sécurité complet (Snyk, SonarQube), communication transparente (changelog)
  \item \textbf{J+7} : Post-mortem rédigé, mesures préventives ajoutées (nouveaux tests, revue processus)
\end{enumerate}

\subsubsection{Perte totale de la base de données}

\textbf{Scénario} : Crash serveur OVH, corruption base PostgreSQL irréparable.

\textbf{Plan d'action} :
\begin{enumerate}
  \item \textbf{Immédiat} : Basculement sur dernière sauvegarde pg\_dump (< 24h de perte)
  \item \textbf{H+1} : Restauration sur nouveau serveur (VPS de secours pré-configuré)
  \item \textbf{H+2} : Vérification intégrité données (checksums, requêtes de validation)
  \item \textbf{H+4} : Remise en production, notification utilisateurs (perte potentielle dernières 24h)
  \item \textbf{J+1} : Migration vers réplication maître-esclave (éviter scénario futur)
\end{enumerate}

\section{Scénarios d'Usage Prioritaires (P1)}

\subsection{Scénario 1 : Création du compte et première connexion}

\subsubsection{Préconditions}
\begin{itemize}
  \item Utilisateur non inscrit, navigateur moderne (Chrome 120+, Firefox 120+, Safari 17+)
  \item Accès internet stable
  \item Email valide et accessible
\end{itemize}

\subsubsection{Étapes du scénario}
\begin{enumerate}
  \item Utilisateur accède à \texttt{https://mysmartbudget.com/auth/register}
  \item Remplit formulaire : nom, prénom, email, mot de passe (12+ caractères, 1 maj, 1 min, 1 chiffre, 1 spécial), confirmation mot de passe
  \item Coche case "J'accepte la politique de confidentialité" (consentement RGPD)
  \item Clique sur "Créer mon compte"
  \item Système valide les données, hache le mot de passe (bcrypt cost 12), crée l'utilisateur en base
  \item Redirection automatique vers \texttt{/dashboard} avec token JWT (expiration 24h)
  \item Dashboard affiche message de bienvenue : "Bienvenue [Prénom] ! Ajoutez votre première transaction"
\end{enumerate}

\subsubsection{Postconditions}
\begin{itemize}
  \item Utilisateur authentifié avec session active (token JWT valide)
  \item Profil créé en base avec rôle "Utilisateur"
  \item 15 catégories prédéfinies associées automatiquement au compte
  \item Log audit : inscription enregistrée (IP, timestamp)
\end{itemize}

\subsubsection{Variantes et exceptions}
\begin{itemize}
  \item \textbf{Email déjà utilisé} : Message d'erreur "Cet email est déjà associé à un compte. Connectez-vous ou réinitialisez votre mot de passe."
  \item \textbf{Mot de passe faible} : Validation temps réel avec indicateur de force (rouge/orange/vert), blocage soumission si critères non respectés
  \item \textbf{Erreur réseau} : Message "Impossible de créer le compte. Vérifiez votre connexion et réessayez." + retry automatique après 3s
\end{itemize}

\subsubsection{Critères de succès}
\begin{itemize}
  \item Temps total du parcours : $< 2$ min (P95)
  \item Taux de réussite : $\geq 95\,\%$ (hors erreurs utilisateur : email doublon, mot de passe faible)
  \item Taux d'activation : $\geq 80\,\%$ (compte créé $\rightarrow$ 1\textsuperscript{re} connexion dans 7j)
\end{itemize}

\subsection{Scénario 2 : Ajout d'une transaction et affectation à une enveloppe}

\subsubsection{Préconditions}
\begin{itemize}
  \item Utilisateur connecté
  \item Au moins 1 enveloppe créée (ex : "Alimentation" avec plafond 400€/mois)
  \item Catégorie "Courses" existante
\end{itemize}

\subsubsection{Étapes du scénario}
\begin{enumerate}
  \item Utilisateur navigue vers \texttt{/transactions}
  \item Clique sur bouton "+ Nouvelle transaction"
  \item Remplit formulaire modal :
    \begin{itemize}
      \item Date : aujourd'hui (pré-rempli, modifiable)
      \item Montant : 35.50€
      \item Type : Dépense (sélecteur radio)
      \item Catégorie : "Courses" (dropdown)
      \item Enveloppe : "Alimentation" (dropdown filtré par catégorie, optionnel)
      \item Libellé : "Supermarché Carrefour" (optionnel)
    \end{itemize}
  \item Clique sur "Enregistrer"
  \item Système valide les données, enregistre la transaction en base (< 500ms)
  \item Mise à jour temps réel :
    \begin{itemize}
      \item Liste des transactions : nouvelle ligne ajoutée en haut (tri date DESC)
      \item Enveloppe "Alimentation" : consommation passe de 150€ à 185.50€ (46\,\% du plafond, indicateur vert)
      \item Dashboard (si ouvert) : KPI "Dépenses du mois" incrémenté de 35.50€
    \end{itemize}
  \item Message de succès (toast vert) : "Transaction ajoutée avec succès"
\end{enumerate}

\subsubsection{Postconditions}
\begin{itemize}
  \item Transaction persistée en base avec \texttt{created\_at = NOW()}
  \item Enveloppe "Alimentation" mise à jour (consommation, pourcentage)
  \item Historique modifiable (bouton "Éditer" visible sur la ligne)
\end{itemize}

\subsubsection{Variantes et exceptions}
\begin{itemize}
  \item \textbf{Montant négatif ou invalide} : Validation temps réel, champ bordure rouge, message "Le montant doit être positif"
  \item \textbf{Date future} : Avertissement "Êtes-vous sûr ? La date est dans le futur" + confirmation requise
  \item \textbf{Enveloppe dépassée} : Si ajout dépasse 100\,\% du plafond, modal d'alerte : "Attention, cette transaction fait dépasser votre enveloppe Alimentation de 15€. Continuer ?" (boutons Oui/Non)
  \item \textbf{Erreur serveur (500)} : Message "Impossible d'enregistrer la transaction. Réessayez." + retry automatique
\end{itemize}

\subsubsection{Critères de succès}
\begin{itemize}
  \item Temps ajout transaction : $< 30$s (P95, incluant saisie utilisateur)
  \item Latence API \texttt{POST /transactions} : $< 500$ms (P95)
  \item Taux d'erreur validation : $< 5\,\%$ (erreurs utilisateur normales : montant invalide, etc.)
  \item Cohérence données : 0 écart entre montant saisi et montant enregistré
\end{itemize}

\subsection{Scénario 3 : Consultation du dashboard et interprétation des indicateurs}

\subsubsection{Préconditions}
\begin{itemize}
  \item Utilisateur connecté avec historique de transactions (minimum 10 transactions sur 2 mois)
  \item Au moins 2 enveloppes créées
\end{itemize}

\subsubsection{Étapes du scénario}
\begin{enumerate}
  \item Utilisateur navigue vers \texttt{/dashboard}
  \item Chargement de la page (< 2s P95) :
    \begin{itemize}
      \item Affichage immédiat du skeleton (cartes grises pulsantes)
      \item Requêtes API parallèles : \texttt{GET /transactions/summary}, \texttt{GET /budgets/summary}
      \item Rendu progressif : KPIs $\rightarrow$ graphiques
    \end{itemize}
  \item KPIs affichés (cartes en haut de page) :
    \begin{itemize}
      \item Solde total : 1\,245€ (icône porte-monnaie, couleur verte si positif)
      \item Dépenses du mois : 890€ (icône flèche bas, couleur rouge)
      \item Revenus du mois : 2\,100€ (icône flèche haut, couleur verte)
      \item Reste à vivre : 355€ (icône tirelire, couleur orange si < 20\,\% revenus)
    \end{itemize}
  \item Graphiques interactifs (section centrale) :
    \begin{itemize}
      \item Camembert : Répartition dépenses par catégorie (top 10, "Autres" agrégé), survol affiche montant exact + pourcentage
      \item Courbe : Évolution dépenses/revenus sur 12 derniers mois (2 lignes, légende interactive), zoom/pan désactivé par défaut
    \end{itemize}
  \item Filtres temporels (haut de page) :
    \begin{itemize}
      \item Boutons radio : "Mois en cours" (sélectionné par défaut), "Trimestre", "Année", "Personnalisé"
      \item Si "Personnalisé" : datepickers (début/fin), bouton "Appliquer"
    \end{itemize}
  \item Utilisateur clique sur "Trimestre" :
    \begin{itemize}
      \item Rechargement des données (< 1s), graphiques mis à jour
      \item KPIs recalculés sur Q4 2025 (octobre--décembre)
    \end{itemize}
  \item Section enveloppes (bas de page) :
    \begin{itemize}
      \item Liste des enveloppes avec barres de progression :
        \begin{itemize}
          \item "Alimentation" : 385€/400€ (96\,\%, barre orange, icône ⚠️)
          \item "Loisirs" : 120€/300€ (40\,\%, barre verte, icône ✓)
          \item "Transport" : 210€/200€ (105\,\%, barre rouge, icône ❌, texte "Dépassement de 10€")
        \end{itemize}
    \end{itemize}
\end{enumerate}

\subsubsection{Postconditions}
\begin{itemize}
  \item Utilisateur a une vision claire de sa situation financière
  \item Identification visuelle immédiate des enveloppes problématiques (orange/rouge)
  \item Données consultées sans modification (lecture seule)
\end{itemize}

\subsubsection{Variantes et exceptions}
\begin{itemize}
  \item \textbf{Aucune transaction} : Message centré "Vous n'avez pas encore de transactions. Ajoutez-en une pour voir vos statistiques." + bouton CTA "+ Ajouter une transaction"
  \item \textbf{Erreur chargement graphiques} : Fallback texte "Impossible de charger les graphiques. Réessayez." + bouton reload
  \item \textbf{Période personnalisée invalide} : Si date début > date fin, message d'erreur "La date de début doit être antérieure à la date de fin"
\end{itemize}

\subsubsection{Critères de succès}
\begin{itemize}
  \item Time-to-Render (TTR) : $< 2$s (P95, mesure Lighthouse)
  \item Taux d'usage hebdomadaire : $> 60\,\%$ des utilisateurs actifs consultent le dashboard au moins 1 fois/semaine
  \item Compréhension des indicateurs : Score SUS $\geq 75$ (question "Je comprends facilement ma situation financière")
\end{itemize}

\section{Plan de Validation Utilisateur}

\subsection{Méthodologie de test}

\subsubsection{Échantillon de testeurs}

\begin{center}
\begin{tabular}{|p{3cm}|p{4cm}|p{6.5cm}|}
\hline
\textbf{Profil} & \textbf{Critères de sélection} & \textbf{Objectifs de test} \\
\hline
Étudiante (Léa) & 18--25 ans, budget serré (< 1000€/mois), usage smartphone prioritaire & Validation scénario 1 et 2, ergonomie mobile, import CSV \\
\hline
Père de famille (Marc) & 35--45 ans, revenus 3000--5000€/mois, gestion enveloppes multiples & Validation scénario 2 et 3, dashboard complexe, export CSV \\
\hline
Senior & 60+ ans, faible appétence numérique, besoin simplicité & Accessibilité (taille texte, contrastes), navigation intuitive \\
\hline
\end{tabular}
\end{center}

\textbf{Nombre de testeurs} : 3--5 personnes (1--2 par profil), recrutement via réseau personnel, associations locales, offre de compensation symbolique (20€ chèque-cadeau Amazon).

\subsubsection{Protocole de test}

\begin{enumerate}
  \item \textbf{Session modérée individuelle (45 min)} :
    \begin{itemize}
      \item Introduction (5 min) : Contexte projet, consentement enregistrement (audio/écran), rappel liberté d'abandon
      \item Think-aloud (30 min) : Testeur verbalise ses actions/pensées pendant 3 scénarios P1
      \item Débriefing (10 min) : Questions ouvertes ("Qu'avez-vous aimé/détesté ?", "Recommanderiez-vous l'appli ?")
    \end{itemize}
  \item \textbf{Questionnaire SUS (5 min)} : 10 questions échelle Likert (1--5), calcul score global (0--100)
  \item \textbf{Mesures objectives} :
    \begin{itemize}
      \item Temps d'accomplissement par tâche (chronomètre)
      \item Nombre d'erreurs (clics incorrects, chemins abandonnés)
      \item Taux de réussite (tâche complétée sans aide / avec aide / échouée)
    \end{itemize}
\end{enumerate}

\subsection{Critères de validation}

\begin{center}
\begin{tabular}{|p{4.5cm}|p{4cm}|p{5cm}|}
\hline
\textbf{Métrique} & \textbf{Seuil acceptable} & \textbf{Seuil cible} \\
\hline
Score SUS global & $\geq 68$ (moyenne industrie) & $\geq 75$ (bon) \\
\hline
Temps scénario 1 (création compte) & $< 3$ min (P95) & $< 2$ min \\
\hline
Temps scénario 2 (ajout transaction) & $< 1$ min (P95) & $< 30$s \\
\hline
Temps scénario 3 (consultation dashboard) & $< 2$ min (P95) & $< 1$ min \\
\hline
Taux de réussite sans aide & $\geq 80\,\%$ & $\geq 90\,\%$ \\
\hline
Taux d'erreur (clics incorrects) & $< 10\,\%$ & $< 5\,\%$ \\
\hline
Recommandation (NPS) & Score $\geq 0$ (neutre) & Score $\geq 50$ (excellent) \\
\hline
\end{tabular}
\end{center}

\subsection{Traçabilité et itération}

\begin{itemize}
  \item \textbf{Documentation} : Compte-rendu par session (verbatims, captures écran, enregistrements), synthèse globale (problèmes récurrents, recommandations)
  \item \textbf{Priorisation problèmes} : Matrice sévérité (bloquant/majeur/mineur) $\times$ fréquence (tous/majorité/rare)
  \item \textbf{Issues GitHub} : 1 issue par problème identifié, étiquette \texttt{usability}, assignation sprint suivant si priorité élevée
  \item \textbf{Retests} : Après corrections majeures, nouvelle session abrégée (15 min, focus problèmes corrigés) avec 1--2 testeurs
\end{itemize}

\section{Mesure de l'Impact par Fonctionnalité (KPIs)}

\subsection{Tableau des KPIs produit}

\begin{center}
\scriptsize
\begin{tabular}{|p{3cm}|p{5.5cm}|p{5cm}|}
\hline
\textbf{Feature} & \textbf{Indicateur} & \textbf{Objectif v1.0 (3 mois post-lancement)} \\
\hline
\textbf{Auth} & Taux d'activation (compte créé $\rightarrow$ 1\textsuperscript{re} connexion 7j) & $\geq 80\,\%$ \\
 & Taux de rétention J+30 & $\geq 40\,\%$ \\
 & Temps moyen inscription & $< 2$ min (P95) \\
\hline
\textbf{Transactions} & Latence création P95 & $< 500$ms \\
 & \# transactions/semaine par utilisateur actif & $\geq 5$ (moyenne) \\
 & Taux d'édition (transactions modifiées / créées) & $< 10\,\%$ (qualité saisie initiale) \\
\hline
\textbf{Enveloppes} & \% utilisateurs avec $\geq 3$ enveloppes actives & $\geq 60\,\%$ \\
 & \% enveloppes dépassées/mois (moyenne utilisateurs) & $< 20\,\%$ (efficacité budgétaire) \\
 & Temps moyen création enveloppe & $< 1$ min \\
\hline
\textbf{Import CSV} & Taux d'import réussi (fichiers valides) & $> 95\,\%$ \\
 & Temps import moyen (10k lignes) & $< 20$s \\
 & \% utilisateurs ayant importé $\geq 1$ fichier & $\geq 30\,\%$ (adoption fonctionnalité) \\
\hline
\textbf{Dashboard} & Time-to-Render (TTR) P95 & $< 2$s \\
 & Taux d'usage hebdomadaire (visites dashboard/semaine) & $> 60\,\%$ \\
 & Taux de rebond (quitte immédiatement) & $< 30\,\%$ \\
 & Durée session moyenne & $> 3$ min (engagement) \\
\hline
\textbf{Export} & \% utilisateurs ayant exporté données (trimestre 1) & $\geq 20\,\%$ \\
 & Temps génération export (10k transactions) & $< 5$s \\
\hline
\end{tabular}
\end{center}

\subsection{Instrumentation technique}

\begin{itemize}
  \item \textbf{Analytics} : Google Analytics 4 (GA4) + événements personnalisés :
    \begin{itemize}
      \item \texttt{user\_signup}, \texttt{user\_login}, \texttt{transaction\_created}, \texttt{envelope\_created}, \texttt{csv\_imported}, \texttt{data\_exported}
    \end{itemize}
  \item \textbf{Performance} : Web Vitals (LCP, FID, CLS) via Next.js Analytics, alertes si dégradation $> 10\,\%$
  \item \textbf{Erreurs} : Sentry (monitoring erreurs front/back), agrégation par type, notification Slack si erreur critique
  \item \textbf{Logs métier} : Winston (back-end), logs structurés JSON :
    \begin{itemize}
      \item \texttt{\{action: "transaction\_created", userId, amount, category, timestamp\}}
    \end{itemize}
  \item \textbf{Dashboard interne} : Metabase (BI tool open-source) connecté à PostgreSQL, requêtes SQL pour KPIs temps réel
\end{itemize}

\section{Définition du MVP (Rappel Synthétique)}

Le \textbf{MVP v1.0} comprend les 5 modules essentiels suivants :

\begin{enumerate}
  \item \textbf{Authentification \& Sessions} : Inscription, connexion JWT, réinitialisation mot de passe, suppression compte (RGPD)
  \item \textbf{Transactions Manuelles} : CRUD complet, catégorisation (15 catégories + perso), filtres/recherche, pagination
  \item \textbf{Enveloppes Budgétaires} : CRUD enveloppes, plafonds mensuels, calcul "reste à vivre", indicateurs visuels (vert/orange/rouge), historique 12 mois
  \item \textbf{Dashboard Synthétique} : 4 KPIs (solde, dépenses, revenus, reste à vivre), 2 graphiques (camembert catégories, courbe évolution), filtres temporels (mois/trim./année)
  \item \textbf{Import/Export CSV} : Import transactions (format gabarit, validation, doublons, rapport), export CSV/JSON (portabilité RGPD)
\end{enumerate}

\textbf{Chaque module} est couvert par :
\begin{itemize}
  \item Critères d'acceptation mesurables (DoD)
  \item Tests automatisés (unitaires + E2E, couverture $\geq 60\,\%$)
  \item Documentation technique (API Swagger, README)
\end{itemize}

\section{Roadmap Jalonnée (Milestones et Dates Estimées)}

\subsection{Planning détaillé par version}

\begin{center}
\scriptsize
\begin{tabular}{|p{2.5cm}|p{2.5cm}|p{8.5cm}|}
\hline
\textbf{Version} & \textbf{Période cible} & \textbf{Jalons / Livrables principaux} \\
\hline
\textbf{v1.0 (MVP)} & Nov. 2025 -- Jan. 2026 & \textbf{M0 (Nov. 2025)} : Schémas Prisma finalisés (MCD/MLD/MPD), API Auth opérationnelle (JWT, bcrypt), CI/CD GitHub Actions configuré \\
 & & \textbf{M1 (Déc. 2025)} : Modules Transactions \& Enveloppes complets (CRUD, tests), Import CSV fonctionnel (validation, rapport) \\
 & & \textbf{M2 (Jan. 2026)} : Dashboard avec graphiques Recharts, tests E2E Playwright (scénarios P1), déploiement staging (Vercel + OVH), tests utilisateurs (3--5 personnes), corrections bugs critiques \\
\hline
\textbf{v1.1} & Février 2026 & Import CSV robuste (encodages multiples, formats dates flexibles), amélioration UX (feedback utilisateurs), premiers KPIs en prod (GA4, Metabase), optimisation performances (lazy loading, pagination) \\
\hline
\textbf{v2.0} & Avril 2026 & Agrégation bancaire POC (Budget Insight API, 3 banques pilotes), notifications push (dépassement enveloppes, objectifs atteints), cache Redis (sessions, compteurs), queue Bull (imports asynchrones), optimisation perfs (P95 dashboard $< 1$s), tests charge (Gatling, 1000 users concurrents) \\
\hline
\end{tabular}
\end{center}

\subsection{Dépendances critiques et risques planning}

\begin{itemize}
  \item \textbf{M0 $\rightarrow$ M1} : Blocage si schémas DB non finalisés (refactoring coûteux ensuite). \textit{Mitigation} : Validation schémas avec pair programming, revue architecture J+7.
  \item \textbf{M1 $\rightarrow$ M2} : Import CSV complexe (formats variés, doublons). \textit{Mitigation} : Spécifications détaillées dès M0, tests avec fichiers réels (10 banques françaises).
  \item \textbf{M2 $\rightarrow$ Livraison} : Tests utilisateurs révèlent bugs majeurs. \textit{Mitigation} : Buffer 2 semaines post-M2, feature flags (rollback rapide si nécessaire).
\end{itemize}

\section{Traçabilité et Responsabilités}

\subsection{Gestion du backlog (GitHub Projects)}

\begin{itemize}
  \item \textbf{Organisation} : Backlog centralisé dans GitHub Projects (vue Kanban), colonnes : Backlog, Todo, In Progress, In Review, Done
  \item \textbf{Issues} : 1 issue par user story ou bug, template standardisé :
    \begin{itemize}
      \item Titre : \texttt{[US-XX] Description courte} ou \texttt{[BUG] Description}
      \item Corps : User story (As a..., I want..., So that...), critères d'acceptation (liste à cocher), effort estimé (points), priorité (P1/P2/P3)
      \item Étiquettes : \texttt{P1}/\texttt{P2}/\texttt{P3}, \texttt{bug}, \texttt{usability}, \texttt{security}, \texttt{performance}, module (\texttt{auth}, \texttt{transactions}, etc.)
    \end{itemize}
  \item \textbf{Milestones} : Alignés à la roadmap (M0, M1, M2, v1.1, v2.0), échéances fermes, % complétion visible
  \item \textbf{Assignation} : Toutes les issues assignées (porteur de projet en v1, équipe en v2+)
\end{itemize}

\subsection{Definition of Done (DoD) globale}

Une fonctionnalité est considérée \textbf{terminée} (Done) si et seulement si :

\begin{enumerate}
  \item \textbf{Critères d'acceptation respectés} : Tous les éléments de la checklist user story validés
  \item \textbf{Tests passants} :
    \begin{itemize}
      \item Tests unitaires : couverture $\geq 60\,\%$ du code ajouté/modifié (Jest)
      \item Tests E2E : scénario principal fonctionnel (Playwright)
      \item Tests manuels : vérification visuelle sur 3 navigateurs (Chrome, Firefox, Safari) et 2 tailles (mobile 375px, desktop 1920px)
    \end{itemize}
  \item \textbf{Documentation mise à jour} :
    \begin{itemize}
      \item API : Swagger/OpenAPI (décorateurs NestJS auto-générés)
      \item Code : Commentaires JSDoc pour fonctions complexes, README module si nécessaire
      \item Utilisateur : Section FAQ/Help Center mise à jour (si nouvelle fonctionnalité majeure)
    \end{itemize}
  \item \textbf{Revue de code effectuée} :
    \begin{itemize}
      \item Pull Request (PR) sur GitHub avec description détaillée (contexte, changements, tests)
      \item Revue par pair (ou auto-revue avec checklist si solo) : lisibilité, sécurité, performances
      \item CI/CD validé : linters (ESLint, Prettier), tests automatisés, build réussi
    \end{itemize}
  \item \textbf{Déployé sur environnement staging} : Validation en conditions réelles (base de données de test, données anonymisées)
  \item \textbf{Pas de régression} : Tests de non-régression (automatisés + manuels) sur fonctionnalités existantes
\end{enumerate}

\subsection{Responsabilités (RACI)}

\begin{center}
\scriptsize
\begin{tabular}{|p{4.5cm}|c|c|c|c|}
\hline
\textbf{Activité} & \textbf{R} & \textbf{A} & \textbf{C} & \textbf{I} \\
\hline
Conception architecture & Porteur & Porteur & Jury (optionnel) & --- \\
\hline
Développement front-end & Porteur & Porteur & --- & --- \\
\hline
Développement back-end & Porteur & Porteur & --- & --- \\
\hline
Modélisation base de données & Porteur & Porteur & Jury (optionnel) & --- \\
\hline
Tests utilisateurs & Porteur & Porteur & Testeurs & --- \\
\hline
Rédaction documentation & Porteur & Porteur & --- & Jury \\
\hline
Déploiement production & Porteur & Porteur & --- & Hébergeur \\
\hline
Validation jalons & Porteur & Jury CDA & Porteur & --- \\
\hline
\end{tabular}
\end{center}

\textbf{Légende RACI} :
\begin{itemize}
  \item \textbf{R (Responsible)} : Réalise l'activité
  \item \textbf{A (Accountable)} : Approuve/valide l'activité (responsable final)
  \item \textbf{C (Consulted)} : Consulté pour avis/expertise
  \item \textbf{I (Informed)} : Informé des résultats
\end{itemize}